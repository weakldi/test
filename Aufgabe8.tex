%!TEX TS-program = pdflatex
%!TEX TS-options = -shell-escape


\newcommand{\obenlinks}{Name der Vorlesung}		% hier Name der Veranstaltung eintragen
\input{config.tex}	% Präambel (ohne die geht nichts!)
\usetikzlibrary{quotes,angles}
\begin{document}
	\section{A8}
	\subsection{b Beweis durch Widerspruch}
	
	\begin{tikzpicture}
		\coordinate (A) at (0,0);
		\coordinate (B) at (0,2);
		\coordinate (C) at (2,2);
		\coordinate (D) at (5,2);
		\coordinate (E) at (5,0);
		\coordinate (F) at (3,0);
		\coordinate (G) at (1.5,3);
	    \coordinate (H) at (3.5,-1);
		\draw  (A) -- (F) -- (E) ;
		\draw (B) node[left] {$c$};
		\draw (B) -- (C) -- (D);
		\draw (A) node[left] {$b$};
		
		\draw (H) -- (F) -- (C) -- (G);
		\draw (H) node[below] {$a$};
		
		\pic [pic text=\text{$\varphi $ = $\epsilon$}, draw, angle radius=1cm,black] {angle=B--C--F};
		\pic [pic text=\text{$\beta$}, draw, angle radius=1cm,black] {angle=F--C--D};
		
		\pic [pic text=$\epsilon$, draw, angle radius=1cm,black] {angle=E--F--C};
		
		\coordinate (I) at (7,0);
		\coordinate (J) at (7,2);
		\coordinate (K) at (11,2);
		\coordinate (L) at (11,0);
		\coordinate (M) at (9,1);
		
		\draw (K) node[right] {$b$};
		\draw (L) node[right] {$c$};
		
		\draw (I) -- (K);
		\draw (J) -- (L);
		
		\node (dot1) at ($(E)!0.5!(I)$) {$...$};
		\node (dot2) at ($(D)!0.5!(J)$) {$...$};
		
		\pic [pic text=$\alpha$, draw, angle radius=1cm,black] {angle=J--M--I};
	\end{tikzpicture}
	\\
	Annahme: $\epsilon = \varphi \Rightarrow$ b und c besitzen einen Schnittpunkt \\
	Wenn ein Schnittpunkt existiert so schließen $\epsilon $ und $\beta$ und $\alpha$ ein Dreieck ein.
	\begin{align}
		\beta &= \pi - \varphi \label{1}\\
		\varphi &= \epsilon \label{2}\\
	\overset{\text{Gleichung }\ref{1} \text{ und }\ref{2}}{\Leftrightarrow}	\beta &= \pi -\epsilon \label{3}\\
	\overset{\text{Dreieck}}{\Rightarrow} \pi &= \alpha + \beta + \epsilon \\
	\Leftrightarrow \alpha &= - \pi + \beta + \epsilon \\
	\overset{\text{Gleichung }\ref{3}}{\Leftrightarrow} \alpha &= - \pi + \pi -\epsilon + \epsilon \\
	\Leftrightarrow \alpha &= 0 
	\end{align}
	$\Rightarrow \text{Da } \alpha \text{ gleich 0 ist exsistiert kein Schnittpunkt!}$
\end{document}
